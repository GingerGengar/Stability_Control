
%Seperator
%Seperator
%Seperator
%Seperator
%Seperator
\section{Tail Sizing Algorithm}
\begin{comment}
\end{comment}

%Seperator
%Seperator
%Seperator
%Seperator
\subsection{Planning}
\begin{comment}
\end{comment}
We are going to claim that the aerodynamic properties of a wing can be completely described if we know the following things:
\begin{enumerate}
\item Lift Slope: How the lift coefficient of wing changes with angle of attack, this is affected by aspect ratio and oswald efficiency factor.
\item Location of aerodynamic center: The non-dimensional location of the wing's neutral point, this is not always quarter chord.
\item Moments about aerodynamic center: The Moment coefficient associated to the wing at its neutral point, affected by camber of wing, but independent to angle of attack
\item Value of lift at Trim AOA: This is affected by initial angle of attack, also due to camber of wing
\end{enumerate}
Before we proceed with dynamics and stability, we have to assume the above are known for all of the lifting surfaces of the aircraft.
Here is a list of equations we need and was derived from the previous section:
$$k_{1,a} = s_{m}c_{hr,a} - x_{ac,a} \quad,\quad k_{1,b} = x_{ac,b} -s_{m}c_{hr,a}$$ 
$$k_{2,u} = m_{f}x_{cg,f} + m_{tb}x_{cg,tb} + m_{tm}x_{cg,tm} \quad,\quad k_{2,l} = m_{f} + m_{tb} + m_{tm}$$ 
$$k_{3,lc} = k_{1,a}k_{2,l} + k_{2,u} \quad,\quad k_{3,ld} = k_{1,a} + x_{cg,wb}$$ 
$$k_{3,rc} = k_{1,b}k_{2,l} - k_{2,u} \quad,\quad k_{3,rd} = k_{1,b} - x_{cg,wb}$$ 
$$k_{4,a} = c_{L\alpha,b}k_{3,rd}c_{bs}$$ 
$$k_{4,b} = c_{L\alpha,b}(k_{3,rc}+k_{3,rd}c_{bf})-A_{w,a}c_{L\alpha,a}k_{3,ld}c_{bs}$$ 
$$k_{4,c} =  - A_{w,a}c_{L\alpha,a}(k_{3,lc}+k_{3,ld}c_{bf})$$ 
$$A_{wb} = \frac{1}{2k_{4,a}}\left(-k_{4,b} \pm \sqrt{k_{4,b}^{2}-4k_{4,a}k_{4,c}}\right)$$
Here is a list of variables that is referred to in the equations above,
\begin{enumerate}
\item $s_{m}$ : Desired Static Margin of the aircraft
\item $c_{hr,a}$ : Chord length of the main wing
\item $x_{ac,a}$ : Aerodynamic Center of main wing
\item $x_{ac,b}$ : Aerodynamic Center of wing B (tail wing)
\item $m_{f}$ : The total mass of the front fuselage
\item $x_{cg,f}$ : Center of Gravity of front fuselage
\item $m_{tb}$ : Mass of the Tail Boom
\item $x_{cg,tb}$ : Location of Center of Gravity of Tail Boom
\item $m_{tm}$ : Mass of the other things on the tail
\item $x_{cg,tm}$ : Location of the center of gravity of the other things on tail
\item $x_{cg,wb}$ : Center of Gravity of wing B (tail wing)
\item $c_{L\alpha,b}$ : Derivative of Lift coefficient with respect to Angle of attack for wing B (Tail Wing)
\item $c_{L\alpha,a}$ : Derivative of Lift coefficient with respect to Angle of attack for wing A (Main Wing)
\item $c_{bf}$ : Fixed Constant for relating Area to Mass of wing B
\item $c_{bs}$ : Fixed Constant for relating Area to Mass of wing B
\item $A_{w,a}$ : Area of wing A (Main Wing)
\item $A_{w,b}$ : Area of wing B (Tail Wing)
\item $k_{ijk}$ : These are all intermediate variables
\end{enumerate}
All of the equations above and the variables above allows us to compute the area of wing B. After we have computed the area of wing B, we need to compute the mass of wing B and the compute the location of the overall center of gravity for the system. After we determine the center of gravity, we can proceed to determining the dynamic pressure experienced by the aircraft (equation $\ref{dynamic pressure at trim}$). After that is done, we can finally compute the lift of both wing A and wing B. These are the relevant equations:
$$m_{wb} = c_{bs}A_{w,b} + c_{bf}$$
$$x_{cg,o} = \frac{ k_{2,u} + m_{wb}x_{cg,wb}}{k_{2,l} + m_{wb}}$$
For our case, $m_{tot} = m_{f} + m_{wb} + m_{tb} + m_{tm}$
$$q = \frac{m_{tot}g}{\displaystyle  A_{w,a}c_{L,0,a} + \frac{A_{w,a}c_{L,0,a}(x_{cg,o}-x_{ac,a}) + A_{w,a}c_{hr,a}c_{M,ac,a} + A_{w,b}c_{hr,b}c_{M,ac,b}}{(x_{ac,b}-x_{cg,o})} } $$
We might want to run iteration loops later. For the computer efficiency in generating an answer, the language of choice is \texttt{Rust}.

%Seperator
%Seperator
%Seperator
%Seperator
\subsection{Common.rs}
\begin{comment}
\end{comment}
This is where common variables like the graviational acceleration and certain common functions like definitions of lift coefficient and dynamic presure get implemented.
\lstinputlisting[language=C]{Tail_Sizing_Algorithm/Common.rs}

%Seperator
%Seperator
%Seperator
%Seperator
\subsection{Main.rs}
\begin{comment}
\end{comment}
This is where the \texttt{Main} function of the program is implemented. The part that uses \texttt{Sizing\_Tail\_Area::default()} essentially uses the default values for the simulation variables. These default values can be overriden from within the main function but we have decided to implement the variables directly as default values. Here we created a \texttt{Sizing\_Tail\_Area} structure and then called methods to operate on it, using the object oriented programming approach.
\lstinputlisting[language=C]{Tail_Sizing_Algorithm/Main.rs} 

%Seperator
%Seperator
%Seperator
%Seperator
\subsection{Tail\_Sizing\_Vars.rs}
\begin{comment}
\end{comment}
This is where all of the data for the analysis on the aircraft tail is defined, and where the default initialization values are also defined. This and the \texttt{Common.rs} part are the only parts in the entire program where variables are initialized. These initial values dictate the entire running and output of the program. \texttt{DL} indicates variables which are later dynamically linked to other variables. Changing these \texttt{DL} variables at the default function is meaningless because these variables are overwritten before they are used later.
\lstinputlisting[language=C]{Tail_Sizing_Algorithm/Tail_Sizing_Vars.rs}

%Seperator
%Seperator
%Seperator
%Seperator
\subsection{Tail\_Area.rs}
\begin{comment}
\end{comment}
These are all the necessary functions needed to compute the tail area of an aircraft given enough information about the "forward fusealge section" of the entire aircraft.
\lstinputlisting[language=C]{Tail_Sizing_Algorithm/Tail_Area.rs}

%Seperator
%Seperator
%Seperator
%Seperator
\subsection{Specific\_Tail\_Optimization.rs}
\begin{comment}
\end{comment}
This is the place where assumptions are made about the system and we start computing the "cost" function which we would want to minimize in a later part.
\lstinputlisting[language=C]{Tail_Sizing_Algorithm/Specific_Tail_Optimization.rs}

%Seperator
%Seperator
%Seperator
%Seperator
\subsection{IO.rs}
\begin{comment}
\end{comment}
This is a file that handles all of the inputs and outputs of the program. This shows all of the variables and data generated which we are interested in.
\lstinputlisting[language=C]{Tail_Sizing_Algorithm/IO.rs}

%Seperator
%Seperator
%Seperator
%Seperator
\subsection{Results}
\begin{comment}
\end{comment}
Compiling, running and then redirecting to a file can be done with the following command,
\begin{lstlisting}[language=Bash]
rustc Main.rs && ./Main > Algo_Res.tex
\end{lstlisting}
$$$$
Here is the result of running the algorithm,
\lstinputlisting{Tail_Sizing_Algorithm/Algo_Res.tex}

\begin{comment}
wing A
dcL/d alpha(deg) = 0.07339710144927537
x_np = 0.232* chord length
x_np = 0.08249920000000001m
cM = -0.13048
\end{comment}

\begin{comment}
c_M vs alpha at center of mass
xplot
range of cg values not a plot just the exist
\end{comment}
