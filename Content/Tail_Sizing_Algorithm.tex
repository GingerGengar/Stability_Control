
%Seperator
%Seperator
%Seperator
%Seperator
%Seperator
\section{Tail Sizing Algorithm}
\begin{comment}
\end{comment}

%Seperator
%Seperator
%Seperator
%Seperator
\subsection{Planning}
\begin{comment}
\end{comment}
We are going to claim that the aerodynamic properties of a wing can be completely described if we know the following things:
\begin{enumerate}
\item Lift Slope: How the lift coefficient of wing changes with angle of attack, this is affected by aspect ratio and oswald efficiency factor.
\item Location of aerodynamic center: The non-dimensional location of the wing's neutral point, this is not always quarter chord.
\item Moments about aerodynamic center: The Moment coefficient associated to the wing at its neutral point, affected by camber of wing, but independent to angle of attack
\item Value of lift at Trim AOA: This is affected by initial angle of attack, also due to camber of wing
\end{enumerate}
Before we proceed with dynamics and stability, we have to assume the above are known for all of the lifting surfaces of the aircraft.
Here is a list of equations we need and was derived from the previous section:
$$k_{1,a} = s_{m}c_{hr,a} - x_{ac,a} \quad,\quad k_{1,b} = x_{ac,b} -s_{m}c_{hr,a}$$ 
$$k_{2,u} = m_{f}x_{cg,f} + m_{tb}x_{cg,tb} + m_{tm}x_{cg,tm} \quad,\quad k_{2,l} = m_{f} + m_{tb} + m_{tm}$$ 
$$k_{3,lc} = k_{1,a}k_{2,l} + k_{2,u} \quad,\quad k_{3,ld} = k_{1,a} + x_{cg,wb}$$ 
$$k_{3,rc} = k_{1,b}k_{2,l} - k_{2,u} \quad,\quad k_{3,rd} = k_{1,b} - x_{cg,wb}$$ 
$$k_{4,a} = c_{L\alpha,b}k_{3,rd}c_{bs}$$ 
$$k_{4,b} = c_{L\alpha,b}(k_{3,rc}+k_{3,rd}c_{bf})-A_{w,a}c_{L\alpha,a}k_{3,ld}c_{bs}$$ 
$$k_{4,c} =  - A_{w,a}c_{L\alpha,a}(k_{3,lc}+k_{3,ld}c_{bf})$$ 
$$A_{wb} = \frac{1}{2k_{4,a}}\left(-k_{4,b} \pm \sqrt{k_{4,b}^{2}-4k_{4,a}k_{4,c}}\right)$$
Here is a list of variables that is referred to in the equations above,
\begin{enumerate}
\item $s_{m}$ : Desired Static Margin of the aircraft
\item $c_{hr,a}$ : Chord length of the main wing
\item $x_{ac,a}$ : Aerodynamic Center of main wing
\item $x_{ac,b}$ : Aerodynamic Center of wing B (tail wing)
\item $m_{f}$ : The total mass of the front fuselage
\item $x_{cg,f}$ : Center of Gravity of front fuselage
\item $m_{tb}$ : Mass of the Tail Boom
\item $x_{cg,tb}$ : Location of Center of Gravity of Tail Boom
\item $m_{tm}$ : Mass of the other things on the tail
\item $x_{cg,tm}$ : Location of the center of gravity of the other things on tail
\item $x_{cg,wb}$ : Center of Gravity of wing B (tail wing)
\item $c_{L\alpha,b}$ : Derivative of Lift coefficient with respect to Angle of attack for wing B (Tail Wing)
\item $c_{L\alpha,a}$ : Derivative of Lift coefficient with respect to Angle of attack for wing A (Main Wing)
\item $c_{bf}$ : Fixed Constant for relating Area to Mass of wing B
\item $c_{bs}$ : Fixed Constant for relating Area to Mass of wing B
\item $A_{w,a}$ : Area of wing A (Main Wing)
\item $A_{w,b}$ : Area of wing B (Tail Wing)
\item $k_{ijk}$ : These are all intermediate variables
\end{enumerate}
All of the equations above and the variables above allows us to compute the area of wing B. After we have computed the area of wing B, we need to compute the mass of wing B and the compute the location of the overall center of gravity for the system.
$$m_{wb} = c_{bs}A_{w,b} + c_{bf}$$
$$x_{cg,o} = \frac{ k_{2,u} + m_{wb}x_{cg,wb}}{k_{2,l} + m_{wb}}$$
For our case, $m_{tot} = m_{f} + m_{wb} + m_{tb} + m_{tm}$


Here are the steps of how to obtain the size of the tail of an aircraft:
\begin{enumerate}
\item Compute value of k's
\item Compute the final value of $A_{w,b}$
\end{enumerate}


